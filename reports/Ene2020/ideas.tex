\documentclass[10pt,letterpaper]{article}
\usepackage{geometry}
 \geometry{
 total={170mm,257mm},
 left=20mm,
 top=20mm,
 right=20mm,
 bottom=20mm,
 }
\usepackage{lipsum}
\usepackage[T1]{fontenc}
\usepackage[utf8]{inputenc}
\usepackage{titlesec}
\usepackage{enumitem}
\usepackage{arev}
\usepackage{cite}
\usepackage{hyperref}
\hypersetup{
    colorlinks=true,
    linkcolor=red,
    filecolor=magenta,      
    urlcolor=red,
}

\setlength{\parindent}{4em}
\setlength{\parskip}{1em}
\renewcommand{\baselinestretch}{2.0}
\newcommand{\ssection}[1]{%
  \section*{\normalfont\scshape #1}}


\title{{\scshape\huge AlgoLook}}
\author{{\scshape\Large Uzmar Gómez}}

\begin{document}

\maketitle

% \ssection{Identification in video}

% \begin{enumerate}
%     \item[] \underline{Principal Component Analysis}

%     This method

%     Advantages:
%     \begin{itemize}
%         \item a
%     \end{itemize}

%     Disadvantages:
%     \begin{itemize}
%         \item a
%     \end{itemize}
% \end{enumerate}

\ssection{Different Arquitectures}

In the article \cite{Tun2019b}, the authors talk on the performance on the detection of introders when using Apache Kafka and Spark. They mention

Also, on the following \href{https://www.infoq.com/articles/video-stream-analytics-opencv/}{link}

\bibliographystyle{unsrt}
\bibliography{library.bib}

\end{document}

%https://www.infoq.com/articles/video-stream-analytics-opencv/
%https://books.google.com.mx/books?id=R46dDwAAQBAJ&pg=PA777&lpg=PA777&dq=facial+recognition+spark&source=bl&ots=0ryABtNfF3&sig=ACfU3U3j847sjjCBd_9_ShihlcwWtM9PKA&hl=es-419&sa=X&ved=2ahUKEwic0KKq_5znAhWGLc0KHW9LALkQ6AEwIHoECAsQAQ#v=onepage&q=facial%20recognition%20spark&f=false